@article{Cournia2020,
abstract = {Virtual high throughput screening (vHTS) in drug discovery is a powerful approach to identify hits: when applied successfully, it can be much faster and cheaper than experimental high-throughput screening approaches. However, mainstream vHTS tools have significant limitations: ligand-based methods depend on knowledge of existing chemical matter, while structure-based tools such as docking involve significant approximations that limit their accuracy. Recent advances in scientific methods coupled with dramatic speedups in computational processing with GPUs make this an opportune time to consider the role of more rigorous methods that could improve the predictive power of vHTS workflows. In this Perspective, we assert that alchemical binding free energy methods using all-atom molecular dynamics simulations have matured to the point where they can be applied in virtual screening campaigns as a final scoring stage to prioritize the top molecules for experimental testing. Specifically, we propose that alchemical absolute binding free energy (ABFE) calculations offer the most direct and computationally efficient approach within a rigorous statistical thermodynamic framework for computing binding energies of diverse molecules, as is required for virtual screening. ABFE calculations are particularly attractive for drug discovery at this point in time, where the confluence of large-scale genomics data and insights from chemical biology have unveiled a large number of promising disease targets for which no small molecule binders are known, precluding ligand-based approaches, and where traditional docking approaches have foundered to find progressible chemical matter.},
author = {Cournia, Zoe and Allen, Bryce K. and Beuming, Thijs and Pearlman, David A. and Radak, Brian K. and Sherman, Woody},
doi = {10.1021/acs.jcim.0c00116},
file = {:C\:/Users/SRV DATOS/AppData/Local/Mendeley Ltd./Mendeley Desktop/Downloaded/Cournia et al. - 2020 - Rigorous free energy simulations in virtual screening.pdf:pdf},
issn = {15205142},
journal = {Journal of Chemical Information and Modeling},
keywords = {absolute,alchemistry,binding,docking,energy,free,methods,mmgbsa,virtual screening},
mendeley-tags = {absolute,alchemistry,binding,docking,energy,free,methods,mmgbsa,virtual screening},
number = {9},
pages = {4153--4169},
pmid = {32539386},
title = {{Rigorous free energy simulations in virtual screening}},
volume = {60},
year = {2020}
}
@article{Yang2010,
abstract = {Pharmacophore approaches have become one of the major tools in drug discovery after the past century's development. Various ligand-based and structure-based methods have been developed for improved pharmacophore modeling and have been successfully and extensively applied in virtual screening, de novo design and lead optimization. Despite these successes, pharmacophore approaches have not reached their expected full capacity, particularly in facing the demand for reducing the current expensive overall cost associated with drug discovery and development. Here, the challenges of pharmacophore modeling and applications in drug discovery are discussed and recent advances and latest developments are described, which provide useful clues to the further development and application of pharmacophore approaches.},
author = {Yang, Sheng-Yong},
doi = {10.1016/j.drudis.2010.03.013},
file = {::},
isbn = {1359-6446},
issn = {13596446},
journal = {Drug discovery today},
keywords = {pharmacophores,review},
mendeley-tags = {pharmacophores,review},
number = {11-12},
pages = {444--450},
pmid = {20362693},
publisher = {Elsevier Ltd},
title = {{Pharmacophore modeling and applications in drug discovery: challenges and recent advances.}},
url = {http://dx.doi.org/10.1016/j.drudis.2010.03.013},
volume = {15},
year = {2010}
}
@article{Schaller2020,
abstract = {3D pharmacophore models are three-dimensional ensembles of chemically defined interactions of a ligand in its bioactive conformation. They represent an elegant way to decipher chemically encoded ligand information and have therefore become a valuable tool in drug design. In this review, we provide an overview on the basic concept of this method and summarize key studies for applying 3D pharmacophore models in virtual screening and mechanistic studies for protein functionality. Moreover, we discuss recent developments in the field. The combination of 3D pharmacophore models with molecular dynamics simulations could be a quantum leap forward since these approaches consider macromolecule–ligand interactions as dynamic and therefore show a physiologically relevant interaction pattern. Other trends include the efficient usage of 3D pharmacophore information in machine learning and artificial intelligence applications or freely accessible web servers for 3D pharmacophore modeling. The recent developments show that 3D pharmacophore modeling is a vibrant field with various applications in drug discovery and beyond. This article is categorized under: Computer and Information Science > Chemoinformatics Computer and Information Science > Computer Algorithms and Programming Molecular and Statistical Mechanics > Molecular Interactions.},
author = {Schaller, David and {\v{S}}ribar, Dora and Noonan, Theresa and Deng, Lihua and Nguyen, Trung Ngoc and Pach, Szymon and Machalz, David and Bermudez, Marcel and Wolber, Gerhard},
doi = {10.1002/wcms.1468},
file = {::},
issn = {17590884},
journal = {Wiley Interdisciplinary Reviews: Computational Molecular Science},
keywords = {3D pharmacophores,artifical intelligence,learning,machine,machine learning,pharmacophores,screening,virtual,virtual screening,web services},
mendeley-tags = {learning,machine,pharmacophores,screening,virtual},
number = {4},
pages = {1--20},
title = {{Next generation 3D pharmacophore modeling}},
volume = {10},
year = {2020}
}
@article{Khedkar2007,
abstract = {Pharmacophore mapping is one of the major elements of drug design in the absence of structural data of the target receptor. The tool initially applied to discovery of lead molecules now extends to lead optimization. Pharmacophores can be used as queries for retrieving potential leads from structural databases (lead discovery), for designing molecules with specific desired attributes (lead optimization), and for assessing similarity and diversity of molecules using pharmacophore fingerprints. It can also be used to align molecules based on the 3D arrangement of chemical features or to develop predictive 3D QSAR models. This review begins with a brief historical overview of the pharmacophore evolution followed by a coverage of the developments in methodologies for pharmacophore identification over the period from inception of the pharmacophore concept to recent developments of the more sophisticated tools such as Catalyst, GASP, and DISCO. In addition, we present some very recent successes of the widely used pharmacophore generation methods in drug discovery.},
author = {Khedkar, Santosh and Malde, Alpeshkumar and Coutinho, Evans and Srivastava, Sudha},
doi = {10.2174/157340607780059521},
file = {::},
issn = {15734064},
journal = {Medicinal Chemistry},
keywords = {database search,fingerprints,modeling,phamacophores,pharmacophore modeling,review,virtual screening},
mendeley-tags = {modeling,phamacophores,review},
number = {2},
pages = {187--197},
pmid = {17348856},
title = {{Pharmacophore Modeling in Drug Discovery and Development: An Overview}},
volume = {3},
year = {2007}
}
@article{Kim2010b,
abstract = {Importance of the field: In research relating to the development of new drugs, hit identification and validations are critical for successful optimization of candidates. To achieve rapid identification of new lead compounds, high-throughput screening assays have been employed in many pharmaceutical companies and laboratories. However, their success depends on the assay system relevant to in vivo conditions and they are physically limited by the repertoire of compounds. As an alternative or complementary approach to high-throughput screening assays, virtual screening is an efficient method to identify drug candidates in silico from large chemical compound databases. Its usefulness has been verified by current applications that successfully retrieved hit and lead identifications against various disease targets. However, for better application, the scoring functions for distinguishing possible active and inactive compounds must beimproved. Areas covered in this review: In this review, we provide an overview of pharmacophore-based virtual screening methods with a special focus on their successful application towards finding hits against various diseasetargets. What the reader will gain: Readers will rapidly gain insight into the recent successful applications of pharmacophore-based virtual screening. They will acknowledge that this technique is a powerful and cost-effective alternative to high-throughput assays. Take home message: Although there are many hurdles yet to be resolved, virtual screening techniques will emerge as essential infrastructure and as a prerequisite for developing new lead compounds with therapeuticapplications. {\textcopyright} 2010 Informa UK Ltd.},
author = {Kim, Kyun Hwan and Kim, Nam Doo and Seong, Baik Lin},
doi = {10.1517/17460441003592072},
file = {::},
isbn = {1746044100359},
issn = {17460441},
journal = {Expert Opinion on Drug Discovery},
keywords = {Anticancer,Antivirus,Drug discovery,Pharmacophore,Virtual screening,pharmacophores,review},
mendeley-tags = {pharmacophores,review},
number = {3},
pages = {205--222},
title = {{Pharmacophore-based virtual screening: A review of recent applications}},
volume = {5},
year = {2010}
}
@article{Carlson2002,
abstract = {The most advanced methods for computer-aided drug design and database mining incorporate protein flexibility. Such techniques are not only needed to obtain proper results; they are also critical for dealing with the growing body of information from structural genomics.},
author = {Carlson, Heather A.},
doi = {10.1016/S1367-5931(02)00341-1},
file = {:D\:/Daniel/Documents/pharmacophore/servicio/Papers/Week1/Carlson2002Protein.pdf:pdf},
issn = {13675931},
journal = {Current Opinion in Chemical Biology},
number = {4},
pages = {447--452},
pmid = {12133719},
title = {{Protein flexibility and drug design: How to hit a moving target}},
volume = {6},
year = {2002}
}
@article{Guner2014,
abstract = {For over a century since the early 1900s, Paul Ehrlich was credited with originating the concept of pharmacophores. This was challenged by John Van Drie in 2007 due to the fact that Ehrlich did not use the word " pharmacophore" in his writings. Van Drie claimed that the attribution of the pharmacophore concept to Ehrlich was due to an erroneous citation made by Ari{\"{e}}ns in a 1966 paper, and instead he claimed, Lemont B. Kier developed the pharmacophore concept (in the modern sense, as defined by the IUPAC) during 1967-1971. There are two separate issues that may have triggered this conflict. The first one is the shift in the meaning of pharmacophore from "chemical groups" to patterns of "abstract features" of a molecule that are responsible for a biological effect. Indeed, the original use of the term is different than the current definition proposed by the IUPAC. The term was redefined in 1960 by Schueler, and this modification formed the basis of IUPAC's modern definition. The second issue is the origin of the "concept" of pharmacophore. While Ehrlich's contemporaries have consistently attributed the origin of the concept to him, the issue is further complicated by the fact that Ehrlich did not use the term pharmacophore in his papers. He, instead, referred to the features of a molecule that are responsible for biological effects as toxophores, while his contemporaries were using the term pharmacophore for the same features. In this paper, we resolve any doubts about the origins of the pharmacophore concept. Our research points to Paul Ehrlich's 1898 paper for originating the concept, which identifies peripheral chemical groups in molecules responsible for binding that leads to the subsequent biological effect, and to Schueler's 1960 book that extends the concept to the modern definition where spatial patterns of abstract features of a molecule define the pharmacophore and are ultimately responsible for the biological effect. {\textcopyright} 2014 American Chemical Society.},
author = {G{\"{u}}ner, Osman F. and Bowen, J. Phillip},
doi = {10.1021/ci5000533},
file = {:D\:/Daniel/Documents/pharmacophore/servicio/Papers/Week1/Guner2014Setting.pdf:pdf},
issn = {15205142},
journal = {Journal of Chemical Information and Modeling},
number = {5},
pages = {1269--1283},
pmid = {24745881},
title = {{Setting the record straight: The origin of the pharmacophore concept}},
volume = {54},
year = {2014}
}
